\section{Introduction}
\label{introduction}


% We discovered a novel approach to create embedded \textbf{D}omain-\textbf{S}pecific \textbf{L}anguages. Our approach is based on two assumptions:
% \begin{enumerate*}
%   \item The return type is the specific domain of DSL.
%   \item A language feature should be adaptive to various domains along with the hosting language control flow.
% \end{enumerate*}
% By combining of the two assumptions, we build a DSL framework, and some examples of DSLs in various domains, including asynchronous or parallel programming, lazy stream generation, resource management.

% A DSL keyword in our framework can be adaptive to its contextual domain. A DSL user can create a single function that contains interleaved DSLs implemented by different vendors, along with ordinary Scala control flows.

% We also provide some adapters to monad, which can be also used as a alternative syntactic sugar of monad comprehension.

DSL operators in our framework can be adaptive to its specific domain. A DSL user can create a single function that contains interleaved different DSLs implemented from different vendors, along with ordinary Scala control flows.

 
In functional programming, imperative control flow can be composed from monads\cite{wadler1990comprehending,wadler1992essence,jones1993composing}.
Along with monad transformers\cite{liang1995monad} and \lstinline{do} or \lstinline{for} comprehension\cite{jones1998haskell,odersky2004scala}, the ability of control flow

We show how a set of building blocks can be used to construct
programming language interpreters, and present implementations
of such building blocks capable of supporting many
commonly known features, including simple expressions,
three different function call mechanisms (call-by-name, call-by-value and lazy evaluation), references and assignment,
nondeterminism, first-class continuations, and program tracing.


\section{Conclusion}
\label{conclusion}


\clearpage
% Appendix
\appendix

\printglossary

\begin{acks}
% TODO:
\end{acks}

% Bibliography
\bibliography{bibliography}
